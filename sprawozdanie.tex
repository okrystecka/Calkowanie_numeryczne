\documentclass{article}
\usepackage{fancyhdr}
%\usepackage{polski}
%\usepackage[cp1250]{inputenc}
\usepackage[T1]{fontenc}
\usepackage[utf8]{inputenc}
\usepackage{lmodern}
\setlength{\textheight}{24.4cm}

\begin{titlepage}
	\title{Sprawozdanie z projektu z Metod Numerycznych}

\end{titlepage}
\begin{document}
		\maketitle	
	\pagestyle{fancy}
	\rhead{Aleksandra Krystecka, 276740\\Mateusz Marczuk, 276748}
	\lhead{Data wykonania}

	
	Sprawozdanie z projektu z Metod Numerycznych
	zawartość...\\
	\begin{center}
		
		\begin{tabular}{|c|c|c|c|c|}
			\hline
			Zad. 1.1 & Zad. 1.2 &  Zad.2 & Zad.3 & Ocena\\\hline
			 & & & & \\
			 & & & & \\\hline
		\end{tabular}

	\end{center}
	\section{Wstep}
	Projekt sprowadza się do obliczenia wartości całki, określonej wzorem (1), używając kwadratur Newtona-Cotesa - metody trapezów i parabol. W celu ułatwienia obliczeń, wyrażenie zostało podzielone na współczynnik $\frac{1}{\sigma\sqrt{2\pi}}$ oraz całkę
		\begin{equation}
			I = \frac{1}{\sigma\sqrt{2\pi}}  \int_{a}^{b} exp  \left( - \frac{(x-\overline{x})^2}{2\sigma^2}   \right) dx
		\end{equation}
	\section{Wyniki}
	A tutaj wrzuce tabelki, wykresy, porownanie wynikow\\
	\begin{tabular}{|c|c|c|c|}
		\hline 
		& \multicolumn{3}{c|}{Przedział całkowania} \\ 
		\hline 
		& $a= \overline{x} - \sigma, b = \overline{x} + \sigma$ & $a= \overline{x} - 2 \sigma, b = \overline{x} + 2\sigma$ & $a= \overline{x} - 3 \sigma, b = \overline{x} + 3 \sigma$ \\ 
		\hline 
		Wynik dokładny & 0.6826894 & 0.95448873 & 0.9973002 \\ 
		\hline 
		Wynik - met. trapezów &  &  &  \\ 
		\hline 
		Błąd met. trapezów &  &  &  \\ 
		\hline 
		Wynik - met. parabol &  &  &  \\ 
		\hline 
		Błąd met. parabol &  &  &  \\ 
		\hline 
	\end{tabular} 
	\section{Wnioski}
	a tu cholera wie co bedzie. Na pewno wydruk napisanego programu xD xD xD
\end{document}
\begin{thebibliography}
	\bibitem{wyklady} Zofia, \textit{Materiały do wykładów oraz projektów}
	
	zawartość...
\end{thebibliography}