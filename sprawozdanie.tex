\documentclass{article}
\usepackage{fancyhdr}
%\usepackage{polski}
%\usepackage[cp1250]{inputenc}
\usepackage[T1]{fontenc}
\usepackage[utf8]{inputenc}
\usepackage{lmodern}
%\setlength{\textheight}{24.4cm}

\begin{titlepage}
	\title{Sprawozdanie z projektu z Metod Numerycznych}

\end{titlepage}
\begin{document}
		\maketitle	
	\pagestyle{fancy}
	\rhead{Aleksandra Krystecka, 276740\\Mateusz Marczuk, 276748}
	\lhead{Data wykonania}

	
	Sprawozdanie z projektu z Metod Numerycznych
	zawartość...\\
	\begin{center}
		
		\begin{tabular}{|c|c|c|c|c|}
			\hline
			Zad. 1.1 & Zad. 1.2 &  Zad.2 & Zad.3 & Ocena\\\hline
			 & & & & \\
			 & & & & \\\hline
		\end{tabular}

	\end{center}
	\section{Wstep}
	Projekt sprowadza się do obliczenia wartości całki, określonej wzorem (1), używając kwadratur Newtona-Cotesa - metody trapezów i parabol. W celu ułatwienia obliczeń, wyrażenie zostało podzielone na współczynnik $\frac{1}{\sigma\sqrt{2\pi}}$ oraz całkę
		\begin{equation}
			I = \frac{1}{\sigma\sqrt{2\pi}}  \int_{a}^{b} exp  \left( - \frac{(x-\overline{x})^2}{2\sigma^2}   \right) dx
		\end{equation}
		Poniżej zamieszczono krótki opis stosowanych metod
		\\\\
		\textbf {Metoda trapezow }
		\\ 
		Metoda opiera się na podziale przedziału całkowania na identyczne odcinki. Następnie dla każdego punktu granicznego odcinka znajduje się wartość funkcji podcałkowej w tym punkcie. Tak powstają trapezy - wysokość jest obranym krokiem całkowania, jedną z podstaw jest wartość funkcji dla jednego końca odcinka, a drugą dla drugiego. Korzystając ze wzoru na pole trapezu i sumując wszystkie powstałe w ten sposób trapezy jesteśmy w stanie podać przybliżoną wartość całki. Naturalnie im mniejszy krok całkowania tym przedział dzielimy na więcej trapezów o mniejszej wysokości, co przekłada się na dokładniejszy wynik. 
		\\\\
		\textbf {Metoda parabol (Simpsona)}
		\\ 
		Metoda parabol jest jedną z najdokładniejszych metod przybliżonego całkowania. Inaczej niż w metodzie trapezów całkę przybliża się sumą wycinków obszarów pod parabolą. Podobnie jak w metodzie trapezów pierwszym krokiem jest podział przedziału całkowania na takie same odcinki. Kolejno oblicza się środek każdego odcinka i wyznacza się wartość funkcji podcałkowej dla tego punktu. Następnie tworzona jest parabola przechodząca przez 3 punkty (wartości funkcji podcałkowej). W ten sposób funkcja zostaje aproksymowana fragmentami parabol. Ponieważ w większości przypadków mamy do czynienia z funkcjami nieliniowymi ta metoda prowadzi do uzyskania najmniejszego błędu.  
		
	\section{Wyniki}
	A tutaj wrzuce tabelki, wykresy, porownanie wynikow\\
	\begin{tabular}{|c|c|c|c|}
		\hline 
		& \multicolumn{3}{c|}{Przedział całkowania} \\ 
		\hline 
		& $a= \overline{x} - \sigma, b = \overline{x} + \sigma$ & $a= \overline{x} - 2 \sigma, b = \overline{x} + 2\sigma$ & $a= \overline{x} - 3 \sigma, b = \overline{x} + 3 \sigma$ \\ 
		\hline 
		Wynik dokładny & 0.6826894 & 0.95448873 & 0.9973002 \\ 
		\hline 
		Wynik - met. trapezów & $0.682386776678695$ & $0.954218606628634$ & $0.997183726883306$ \\ 
		\hline 
		Błąd względny met. trapezów &  &  &  \\ 
		\hline 
		Wynik - met. parabol & $0.682083809168075$ & $0.954499438241541$ & $0.997061231960167$ \\ 
		\hline 
		Błąd względny met. parabol &  &  &  \\ 
		\hline 
	\end{tabular} 
\\ \\Ponadto, oszacowany został błąd obu metod na zadanych przedziałach. Określony on został dla metody trapezów jako:
\begin{equation}
E \left(f\right) = - \frac{\left(b-a\right)^{3}}{12n^{2}} f^{\left(2\right)} \left( \xi \right)
\end{equation}
gdzie $\xi \in \left[a, b\right]$, oraz dla metody parabol:
\begin{equation}
E \left(f\right) = - \frac{\left(b-a\right)^{5}}{180n^{4}} f^{\left(4\right)} \left( \xi \right)
\end{equation}
gdzie $\xi \in \left[a, b\right]$. \\ Możemy zauważyć, że jest on zależny od odpowiednio 3 i 5 potęgi przedziału całkowania. Można więc go zmniejszyć, zmniejszając dany przedział. Widoczne jest to w wynikach obliczeń, zawartych w poniższej tabeli: \\
\begin{tabular}{|p{3.5cm}|c|c|c|}
	\hline 
	& \multicolumn{3}{c|}{Przedział całkowania} \\ 
	\hline 
	& $a= \overline{x} - \sigma, b = \overline{x} + \sigma$ & $a= \overline{x} - 2 \sigma, b = \overline{x} + 2\sigma$ & $a= \overline{x} - 3 \sigma, b = \overline{x} + 3 \sigma$ \\ 
	\hline 
	Oszacowanie błędu met. trapezów & -1.0417e-08 & -2.0833e-04 & -1.2500e-05  \\ 
	\hline 
	Oszacowanie błędu met. parabol & -3.2552e-15 & -6.5104e-07 & -1.5625e-09 \\ 
	\hline 
\end{tabular} 


	\section{Wnioski}
	a tu cholera wie co bedzie. Na pewno wydruk napisanego programu xD xD xD
\end{document}
\begin{thebibliography}
	\bibitem{wyklady} Zofia, \textit{Materiały do wykładów oraz projektów}
	
	zawartość...
\end{thebibliography}