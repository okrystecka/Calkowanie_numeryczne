\documentclass{article}
\usepackage{fancyhdr}
%\usepackage{polski}
%\usepackage[cp1250]{inputenc}
\setlength{\textheight}{24.4cm}
\usepackage[T1]{fontenc}
\usepackage[utf8]{inputenc}
\usepackage{lmodern}


\begin{titlepage}
	\title{Sprawozdanie z projektu z Metod Numerycznych}

\end{titlepage}
\begin{document}
		\maketitle	
	\pagestyle{fancy}
	\rhead{Aleksandra Krystecka, 276740\\Mateusz Marczuk, 276748}
	\lhead{Data wykonania}

	
	Sprawozdanie z projektu z Metod Numerycznych
	zawartość...\\
	\begin{center}
		
		\begin{tabular}{|c|c|c|c|c|}
			\hline
			Zad. 1.1 & Zad. 1.2 &  Zad.2 & Zad.3 & Ocena\\\hline
			 & & & & \\
			 & & & & \\\hline
		\end{tabular}

	\end{center}
	\section{Wstep}
	Projekt sprowadza się do obliczenia wartości całki, określonej wzorem (1), używając kwadratur Newtona-Cotesa - metody trapezów i parabol. W celu ułatwienia obliczeń, wyrażenie zostało podzielone na współczynnik $\frac{1}{\sigma\sqrt{2\pi}}$ oraz całkę
		\begin{equation}
			I = \int_{a}^{b} exp  \left( - \frac{(x-\overline{x})^2}{2\sigma^2}   \right) dx
		\end{equation}
	Poniżej zamieszczono krótki opis stosowanych metod
	\\\\
	\textbf {Metoda trapezow }
	\\ 
	Metoda opiera się na podziale przedziału całkowania na identyczne odcinki. Następnie dla każdego punktu granicznego odcinka znajduje się wartość funkcji podcałkowej w tym punkcie. Tak powstają trapezy - wysokość jest obranym krokiem całkowania, jedną z podstaw jest wartość funkcji dla jednego końca odcinka, a drugą dla drugiego. Korzystając ze wzoru na pole trapezu i sumując wszystkie powstałe w ten sposób trapezy jesteśmy w stanie podać przybliżoną wartość całki. Naturalnie im mniejszy krok całkowania tym przedział dzielimy na więcej trapezów o mniejszej wysokości, co przekłada się na dokładniejszy wynik. 
	\\\\
	\textbf {Metoda parabol (Simpsona)}
	\\ 
	Metoda parabol jest jedną z najdokładniejszych metod przybliżonego całkowania. Inaczej niż w metodzie trapezów całkę przybliża się sumą wycinków obszarów pod parabolą. Podobnie jak w metodzie trapezów pierwszym krokiem jest podział przedziału całkowania na takie same odcinki. Kolejno oblicza się środek każdego odcinka i wyznacza się wartość funkcji podcałkowej dla tego punktu. Następnie tworzona jest parabola przechodząca przez 3 punkty (wartości funkcji podcałkowej). W ten sposób funkcja zostaje aproksymowana fragmentami parabol. Ponieważ w większości przypadków mamy do czynienia z funkcjami nieliniowymi ta metoda prowadzi do uzyskania najmniejszego błędu.  
	
	\section{Wyniki}
	A tutaj wrzuce tabelki, wykresy, porownanie wynikow
	\section{Wnioski}
	a tu cholera wie co bedzie. Na pewno wydruk napisanego programu xD xD xD
\end{document}
\begin{thebibliography}
	\bibitem{wyklady} Zofia, \textit{Materiały do wykładów oraz projektów}
	
	zawartość...
\end{thebibliography}