\documentclass{article}
\usepackage{fancyhdr}
%\usepackage{polski}
%\usepackage[cp1250]{inputenc}
\usepackage[T1]{fontenc}
\usepackage[utf8]{inputenc}
\usepackage{lmodern}
%\setlength{\textheight}{24.4cm}

\begin{titlepage}
	\title{Sprawozdanie z projektu z Metod Numerycznych}

\end{titlepage}
\begin{document}
		\maketitle	
	\pagestyle{fancy}
	\rhead{Aleksandra Krystecka, 276740\\Mateusz Marczuk, 276748}
	\lhead{Data wykonania}

	
	\textbf{Temat nr 4: Całkowanie numeryczne przy użyciu kwadratur Newtona-Cotesa}\\
	\begin{center}
		
		\begin{tabular}{|c|c|c|c|c|}
			\hline
			Zad. 1.1 & Zad. 1.2 &  Zad.2 & Zad.3 & Ocena\\\hline
			 & & & & \\
			 & & & & \\\hline
		\end{tabular}

	\end{center}
	\section{Wstep}
	Projekt sprowadza się do obliczenia wartości całki, określonej wzorem (1), używając kwadratur Newtona-Cotesa - metody trapezów i parabol. W celu ułatwienia obliczeń, wyrażenie zostało podzielone na współczynnik $\frac{1}{\sigma\sqrt{2\pi}}$ oraz całkę
		\begin{equation}
			I = \frac{1}{\sigma\sqrt{2\pi}}  \int_{a}^{b} exp  \left( - \frac{(x-\overline{x})^2}{2\sigma^2}   \right) dx
		\end{equation}
		Poniżej zamieszczono krótki opis stosowanych metod
		\\\\
		\textbf {Metoda trapezow }
		\\ 
		Metoda opiera się na podziale przedziału całkowania na identyczne odcinki. Następnie dla każdego punktu granicznego odcinka znajduje się wartość funkcji podcałkowej w tym punkcie. Tak powstają trapezy - wysokość jest obranym krokiem całkowania, jedną z podstaw jest wartość funkcji dla jednego końca odcinka, a drugą dla drugiego. Korzystając ze wzoru na pole trapezu i sumując wszystkie powstałe w ten sposób trapezy jesteśmy w stanie podać przybliżoną wartość całki. Poniżej przedstawiono wartość kwadratury na przedziale [a,b] dla funkcji f(x).
		
		\begin{equation}	
		K_{n}^{T}=\frac{h}{2}\sum_{i=0}^{n-1} \left(f(a)+f(b)\right)  , f_{i} = f\left(x_{i}\right)
		\end{equation}
		gdzie h - długość przedziału całkowania.\\
		Naturalnie im mniejszy krok całkowania tym przedział dzielimy na więcej trapezów o mniejszej wysokości, co przekłada się na dokładniejszy wynik. 
		\\\\
		\textbf {Metoda parabol (Simpsona)}
		\\ 
		Metoda parabol jest jedną z najdokładniejszych metod przybliżonego całkowania. Inaczej niż w metodzie trapezów całkę przybliża się sumą wycinków obszarów pod parabolą. Podobnie jak w metodzie trapezów pierwszym krokiem jest podział przedziału całkowania na takie same odcinki. Kolejno oblicza się środek każdego odcinka i wyznacza się wartość funkcji podcałkowej dla tego punktu. Następnie obliczane są współczynniki paraboli $g_{i}$(x) przechodzącej przez 3 punkty (wartości funkcji podcałkowej) postaci
		\begin{center}$g_{i}(x)=a_{i}x^{2}+b_{i}x+c_{i} $ gdzie $i\epsilon<x_{n},x_{n+1}>$ \end{center}
		Wzór Simpsona na przedziale [a,b]
		\begin{equation}
		K_{n}^{P}(f)=\frac{h}{3}\sum_{i=0}^{n/2-1}(f_{2i}+4f_{2i+1}+f_{2i+2}), f_{i} = f\left(x_{i}\right) 
		\end{equation}
		W ten sposób funkcja zostaje aproksymowana fragmentami parabol. Ponieważ w większości przypadków mamy do czynienia z funkcjami nieliniowymi ta metoda prowadzi do uzyskania najmniejszego błędu.  
		
		
		Przeprowadzono symulacje podanych metod w środowisku MatLab. Zbadano dla każdej metody wynik obliczonej całki a także błędy związane z zastosowaniem metod numerycznych dla 3 przedziałów całkowania. Dobrano także krok całkowania aby 
		
	\section{Wyniki}
Wyniki obliczeń zostały zebrane w poniższej tabeli:\\
	\begin{tabular}{|c|c|c|c|}
		\hline 
		& \multicolumn{3}{c|}{Przedział całkowania} \\ 
		\hline 
		& $a= \overline{x} - \sigma, b = \overline{x} + \sigma$ & $a= \overline{x} - 2 \sigma, b = \overline{x} + 2\sigma$ & $a= \overline{x} - 3 \sigma, b = \overline{x} + 3 \sigma$ \\ 
		\hline 
		Wynik dokładny & 0.6826894 & 0.95448873 & 0.9973002 \\ 
		\hline 
		Wynik - met. trapezów & $0.682386776678695$ & $0.954218606628634$ & $0.997183726883306$ \\ 
		\hline 
		Błąd względny met. trapezów$\left[\%\right]$ & $0.04433$ & $0.02829$ & $0.01168$  \\ 
		\hline 
		Wynik - met. parabol & $0.682083809168075$ & $0.954499438241541$ & $0.997061231960167$ \\ 
		\hline 
		Błąd względny met. parabol $\left[\%\right]$ & $0.08870$ & $0,0012$ & $0.02396$ \\ 
		\hline 
	\end{tabular} 
\\ gdzie błąd względny został wyliczony ze wzoru: $\frac{licznik}{mianownik} $ \\Ponadto, oszacowany został błąd obu metod na zadanych przedziałach. Określony on został dla metody trapezów jako:
\begin{equation}
E \left(f\right) = - \frac{\left(b-a\right)^{3}}{12n^{2}} f^{\left(2\right)} \left( \xi \right)
\end{equation}
gdzie $\xi \in \left[a, b\right]$, oraz dla metody parabol:
\begin{equation}
E \left(f\right) = - \frac{\left(b-a\right)^{5}}{180n^{4}} f^{\left(4\right)} \left( \xi \right)
\end{equation}
gdzie $\xi \in \left[a, b\right]$. \\ Możemy zauważyć, że jest on zależny od odpowiednio 3 i 5 potęgi przedziału całkowania. Można więc go zmniejszyć, zmniejszając dany przedział. Widoczne jest to w wynikach obliczeń, zawartych w poniższej tabeli: \\
\begin{tabular}{|p{3.5cm}|c|c|c|}
	\hline 
	& \multicolumn{3}{c|}{Przedział całkowania} \\ 
	\hline 
	& $a= \overline{x} - \sigma, b = \overline{x} + \sigma$ & $a= \overline{x} - 2 \sigma, b = \overline{x} + 2\sigma$ & $a= \overline{x} - 3 \sigma, b = \overline{x} + 3 \sigma$ \\ 
	\hline 
	Oszacowanie błędu met. trapezów & -1.0417e-08 & -2.0833e-04 & -1.2500e-05  \\ 
	\hline 
	Oszacowanie błędu met. parabol & -3.2552e-15 & -6.5104e-07 & -1.5625e-09 \\ 
	\hline 
\end{tabular} 

Narysowano także wykresy błędów metody, w zależności od kroku całkowania:
%\includegraphics[width=\textwidth]{Trap1.jpg}
%\includegraphics[width=\textwidth]{Trap2.jpg}
%\includegraphics[width=\textwidth]{Trap3.jpg}
%\includegraphics[width=\textwidth]{Para1.jpg}
%\includegraphics[width=\textwidth]{Para2.jpg}
%\includegraphics[width=\textwidth]{Para3.jpg}
	\section{Wnioski}
	Otrzymane wyniki dowiodły możliwości przeprowadzenia całkowania numerycznego z wykorzystaniem metod trapezów i parabol. W obu przypadkach otrzymane wyniki są zgodne (do kilku miejsc po przecinku) co do podanego wyniku dokładnego. Jak wszystkie metody numeryczne również i te nie są pozbawione błędów wynikających przede wszystkim z przybliżeń.\\
	Zauważono wpływ przedziału całkowania na efektywnośc obu metod, ponadto bardzo istotnym czynnikiem był też krok całkowania, który należało dobrać tak aby uzyskać wynik całki podobny do podanego.\\
	Sporządzone wykresy błędu  obrazują w jaki sposób zmienia się błąd w zależności od obranego kroku całkowania\\
\end{document}
\begin{thebibliography}
	\bibitem{wyklady} Zofia, \textit{Materiały do wykładów oraz projektów}
	
	zawartość...
\end{thebibliography}